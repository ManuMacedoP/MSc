\documentclass[12pt]{article}
\usepackage[spanish]{babel}
\usepackage[left=2.5cm,top=2.5cm,right=2.5cm,bottom=2.5cm]{geometry}
\usepackage{amsthm,amssymb,amsmath}
\usepackage[spanish]{cleveref}
\usepackage{bbm}
\usepackage[utf8]{inputenc}
\usepackage{mathrsfs}
\usepackage{enumitem}
\usepackage{multirow}
\usepackage{parskip}
\usepackage{graphicx}
\usepackage{booktabs}
\usepackage{float}
\usepackage{longtable}
\usepackage{algorithm}
\usepackage{algpseudocode}
\usepackage{subfig}
\usepackage{appendix}
\usepackage{lscape}


\renewcommand{\baselinestretch}{1.5}
\setlength{\parindent}{12pt}


\begin{document}
\begin{titlepage}

        \centering
        {\bfseries \ }
        \vspace{2cm}
        
        {\includegraphics[width=0.9\textwidth]{pics/ucm_logo.png}} 
        \vspace{0.6cm}
        
        {\bfseries \Large TRABAJO DE BASES DE DATOS SQL}
        \vspace{1.75cm}
        
        {\bfseries \LARGE Manuel Macedo Pulido}
        \vspace{6mm} 

        {\large Máster Big Data, Data Science e Inteligencia Artificial}
        {\large 2025 }
        \vspace{6mm}

        \thispagestyle{empty}

    \end{titlepage}

    \pagenumbering{arabic} 
    \setcounter{page}{1}

    \newpage
    \tableofcontents

    \newpage
    \begin{landscape}

        \section{Diagrama Entidad-Relación}
        {\includegraphics[width=1.4\textwidth]{pics/diagrama.png}} 

    \end{landscape}

    \newpage
    \section{Diseño conceptual}
    En esta sección se definen y explican las entidades del modelo 'Entidad-Relación' para el ejercicio \textit{Empresa de eventos}, 
    así como sus relaciones y las asunciones que se han considerado para construir el modelo. 

    \subsection{Entidades}
    Entidades seleccionadas para el diseño conceptual:

    \begin{itemize}

        \item \textbf{Evento}\\
            Esta entidad es el centro de todo el ejercicio, pues representa los diferentes eventos que la empresa organiza.
            Y cuenta con los siguientes atributos: \\
                \hspace{0.5cm} -- {\bfseries Nombre Ev}. \\
                \hspace{0.5cm} -- {\bfseries Fecha}: será un datetime (e.g. 1999-09-26 14:30:45). \\
                \hspace{0.5cm} -- {\bfseries Precio}: se asume que es fijo para cada evento, pues no se indica lo contrario en el enunciado. 
                Además, por no complicar demasiado el problema, asumimos que cada persona solo puede comprar su propia entrada (de este modo evitamos 
                que se produzcan las reventas de entradas masivas, como pasa hoy en día), que no hay reembolsos y que no se almacenan datos de las compras de 
                las entradas, de modo que las ganancias de cada evento se podrán calcular mediante la fórmula 
                'ganancia = (número de asistentes * precio entrada) - (suma de cachés + precio alquiler)', asumiendo que todo asistente debe pagar la entrada 
                para acceder al concierto. \\
                \hspace{0.5cm} -- {\bfseries Descripción}.

        \item \textbf{Actividad}\\
            Esta entidad representa las actividades que se realizan en cada evento. 
            Y cuenta con los siguientes atributos:\\
                \hspace{0.5cm} -- {\bfseries Nombre Ac}: se asume que el nombre de la actividad es único (e.g. A night of summer magic) para cada actividad. \\
                \hspace{0.5cm} -- {\bfseries Tipo Ac}: por simplicidad, únicamente podrá tener los valores especificados en el enunciado (conciertos de música, 
                exposiciones, obras de teatro y conferencias) y no se considerarán los géneros de los conciertos de música.

        \item \textbf{Artista}\\
            Esta entidad representa aquellos artistas que participan en los eventos. 
            Y cuenta con los siguientes atributos: \\
                \hspace{0.5cm} -- {\bfseries Nombre Ar}: se asume que el nombre del artista es único (e.g. Leiva). \\
                \hspace{0.5cm} -- {\bfseries Biografía}.

        \item \textbf{Persona}\\
            La entidad persona se puede configurar de diferentes formas. En este caso, se decide seguir la siguiente lógica:
            asumiendo que puede haber personas que tengan mismo nombre y apellidos (e.g. hermanos que se llamen igual o incluso 
            desconocidos que se llamen exactamente igual), y considerando que en el enunciado dice "teléfonos de contacto" (asumimos que como máximo se 
            aceptan dos, pues habitualmente suele ser el teléfono móvil y el fijo),
            asumimos que toda persona que se registre en un evento tendrá que tener un mail, que además será único. 
            Por último, se tendrá en cuenta que el número de personas que asista a cada evento, deberá cumplir la restricción de aforo de cada evento 
            (número de asistentes del evento $\leq$ aforo de su ubicación).
            Por lo que cuenta con los siguientes atributos: \\
                \hspace{0.5cm} -- {\bfseries Nombre Pe}: para no complicar demasiado el problema, asumimos que el nombre completo se introduce en realidad en 3 campos. \\
                \hspace{0.5cm} -- {\bfseries Apellido1}. \\
                \hspace{0.5cm} -- {\bfseries Apellido2}. \\
                \hspace{0.5cm} -- {\bfseries Teléfono1}. \\
                \hspace{0.5cm} -- {\bfseries Teléfono2}. \\
                \hspace{0.5cm} -- {\bfseries Mail}.
        
        \item \textbf{Ubicación}\\
            Por último, se define la entidad ubicación, que tiene algo más de detalle. El nombre de la ubicación puede no ser único, pues podemos tener el mismo
            nombre en diferentes localidades. Es por ello que necesitamos saber de manera independiente tanto la ubicación como la localidad.
            Además, el atributo localidad es compuesto, por lo que creará tabla. E inicialmente, podríamos pensar que sería suficiente con usar el tipo de localidad 
            y su nombre. Sin embargo, en caso de que nos equivocásemos al introducir el valor del nombre en la tabla localidad, también tendríamos que corregirlo
            en la tabla de la entidad ubicación. Por ello, para evitar este tipo de problemas, añadimos el atributo ID localidad, que tendrá valores únicos, 
            y asumimos que no se cometerán fallos al añadir campos nuevos, pues será un ID numérico al uso. 
            Y cuenta con los siguientes atributos: \\
                \hspace{0.5cm} -- {\bfseries Nombre Ub}. \\
                \hspace{0.5cm} -- {\bfseries Localidad}: este a su vez tiene 3 atributos (ID Localidad, Nombre Localidad (que únicamente contiene el 
                topónimo) y Tipo Lo (que podrá tener únicamente el valor 'ciudad' o el valor 'pueblo')). \\
                \hspace{0.5cm} -- {\bfseries Dirección}: asumimos que la dirección será un único valor que contendrá toda la información necesaria,
                referente a la localidad previamente especificada. \\
                \hspace{0.5cm} -- {\bfseries Aforo}: este atributo será utilizado para tener en cuenta la restricción de aforo. \\
                \hspace{0.5cm} -- {\bfseries Precio Alquiler}. \\
                \hspace{0.5cm} -- {\bfseries Características}: para no complicar demasiado el ejercicio, se asume que este atributo será una cadena de texto 
                (también podría representarse mediante un atributo multivariado, indicado con doble círculo en el diagrama), por lo que no habrá consultas 
                estructuradas por características.

    \end{itemize}

    \newpage
    \subsection{Relaciones}
    A continuación se explican las diferentes relaciones que tienen lugar entre las entidades:

    \begin{itemize}

        \item \textbf{Cuesta} -- Relación entre Actividad y Artista\\
            Esta relación es de tipo N:N (por lo que creará tabla), pues varios artistas pueden participar en varias actividades, y a partir de ella se 
            podrán calcular los costes totales por actividad. Además se considera que el mismo artista siempre tendrá el mismo coste para la misma actividad 
            aunque esta tenga lugar en diferentes eventos. \\
            La cardinalidad del lado Actividad → Artista es de tipo (1,N), pues el enunciado dice que 'En cada actividad participa uno o varios artistas". \\
            Y la cardinalidad del lado Artista → Actividad es de tipo (1,N), pues el enunciado dice 'de los artistas que los protagonizan', por lo que asumimos
            que solo se registran artistas que van a participar en actividades (y no artistas que no participan en actividades, lo que daría una cardinalidad
            de tipo (0,N)). \\
            Además, tiene el atributo de relación 'dinero', pues entendemos que cada artista tiene un coste diferente para cada actividad.
        
        \item \textbf{Se Realiza} -- Relación entre Evento y Actividad\\
            Esta relación es de tipo 1:N, pues el enunciado dice que 'En un evento solo se realiza una actividad', pero no explicita que una actividad solo se 
            realice en un evento. \\
            La cardinalidad del lado Evento → Actividad es de tipo (1,1), pues en un evento solo puede haber una actividad. \\
            Y la cardinalidad del lado Actividad → Evento es de tipo (1,N), una misma actividad puede realizarse en varios eventos.

        \item \textbf{Asiste} -- Relación entre Evento y Persona\\
            Esta relación es de tipo N:N (por lo que creará tabla), pues varias personas pueden asistir a varios eventos. \\
            La cardinalidad del lado Evento → Persona es de tipo (0,N), pues el enunciado no dice explícitamene que a cada evento tengan que asistir al menos
            una persona. \\
            Y la cardinalidad del lado Persona → Evento es de tipo (1,N), pues el enunciado dice 'tendremos en cuenta los asistentes a los eventos', 
            por lo que asumimos que solo se registran personas que van a asistir a eventos (y no personas que no asisten a eventos, lo que daría una cardinalidad
            de tipo (0,N)). \\
            Además, tiene el atributo de relación 'valoración' (se asume que solo podrán valorar aquellas personas que asisten a un evento, 
            y se comprobará que esté entre 0 y 5 (0 $\leq$ valoración $\leq$ 5), permitiendo NULL hasta que el asistente valore), ya que la valoración 
            de cada persona a cada evento es independiente.
                
        \item \textbf{Tiene} -- Relación entre Evento y Ubicación\\
            Esta relación es de tipo 1:N, asumiendo que cada evento esté solo en 1 ubicación (e.g. no hay eventos en el medio de 2 ubicaciones)
            y asumiendo que una ubicación puede tener más de 1 evento. \\
            La cardinalidad del lado Evento → Ubicación es de tipo (1,1), pues asumimos que cada evento tiene 1 ubicación, y que es única. \\
            Y la cardinalidad del lado Ubicación → Evento es de tipo (1,N), ya que según el enunciado, asumimos que solo se registran ubicaciones asociadas 
            a eventos (en caso contrario la cardinalidad sería de tipo (0,N)) y que varios eventos pueden tener lugar en una misma ubicación.

    \end{itemize}

    \newpage
    \subsection{Paso a tabla}
    A continuación se detallan las diferentes tablas que se van a necesitar para representar el diagrama anteriormente descrito, así como las principales columnas 
    (Primary Key (PK), Foreign Key (FK) y Relation Attribute):

    \begin{itemize}

        \item \textbf{Tabla de la entidad 'Actividad'} \\
            -- PK: Nombre Ac.
        
        \item \textbf{Tabla de la entidad 'Artista'} \\
            -- PK: Nombre Ar.
            
        \item \textbf{Tabla de la entidad 'Persona'} \\
            -- PK: Mail.
            
        \item \textbf{Tabla del atributo compuesto 'Localidad'} \\
            -- PK: ID Localidad.

        \item \textbf{Tabla de la entidad 'Ubicación'} \\
            -- PK: (compuesta): Nombre Ub, ID Localidad (ya que como se indicó previamente, para que la ubicación sea única, se necesita su nombre y 
            el ID de la localidad donde se encuentra). \\
            -- FK: ID Localidad (ya que con la PK se identifica unívocamente la ubicación, pero además se necesita saber cual es el id de la localidad 
            para garantizar que ese id exista en la tabla localidad).

        \item \textbf{Tabla de la entidad 'Evento'} \\
            -- PK (compuesta): Nombre Ev, Fecha (ya que asumimos que un mismo evento se puede realizar en diferentes fechas). \\
            -- FK: Nombre Ac (para asegurar que toda fila de 'Evento' apunte a una actividad existente). \\
            -- FK (compuesta): Nombre Ub, ID Localidad (para que apunte a una Candidate Key (CK) única).
            
        \item \textbf{Tabla de la relación 'Cuesta'} \\
            -- PK (compuesta): Nombre Ac, Nombre Ar (ya que la relación 'Cuesta' es de tipo N:N, entonces su PK está formada por las dos PK de las 
            entidades que relaciona). \\
            -- FK: Nombre Ac (para asegurar que toda fila de 'Cuesta' apunte a una actividad existente). \\
            -- FK: Nombre Ar (para asegurar que toda fila de 'Cuesta' apunte a un artista existente). \\
            -- Relation Attribute: Dinero.
            
        \item \textbf{Tabla de la relación 'Asiste'} \\
            -- PK (compuesta): Nombre Ev, Fecha, Mail (ya que la relación 'Asiste' es de tipo N:N, entonces su PK está formada por las dos PK de las 
            entidades que relaciona). \\
            -- FK (compuesta): Nombre Ev, Fecha (para asegurar que toda fila de 'Asiste' apunte a un evento existente). \\
            -- FK: Mail (para asegurar que toda fila de 'Asiste' apunte a una persona existente). \\
            -- Relation Attribute: Valoración.

    \end{itemize}

    \newpage
    \section{Implementación}
    En esta última sección, se explica en detalle el código a utilizar para resolver el problema del enunciado. 
    
    \subsection{Creación de la Base de Datos}    
    En primer lugar, se introducen la información de la base de datos (autor y nombre), y se genera la base de datos:

    \begin{center}{\includegraphics[width=0.8\textwidth]{pics/0.crear.png}}\end{center} 

    Seguidamente, se crean las tablas siguiendo un orden topológico basado en las dependencias de FK, 
    comenzando por las tablas independientes y progresando hacia las dependientes:
    
    \begin{itemize}
        \item \textbf{Tabla Actividad}
        \begin{center}{\includegraphics[width=0.8\textwidth]{pics/1.1.actividad.png}}\end{center}
        \item \textbf{Tabla Artista}
        \begin{center}{\includegraphics[width=0.8\textwidth]{pics/1.2.artista.png}}\end{center}
        \item \textbf{Tabla Persona}
        \begin{center}{\includegraphics[width=0.8\textwidth]{pics/1.3.persona.png}}\end{center}
        \item \textbf{Tabla Localidad}
        \begin{center}{\includegraphics[width=0.8\textwidth]{pics/1.4.localidad.png}}\end{center}
        \item \textbf{Tabla Ubicación}
        \begin{center}{\includegraphics[width=0.8\textwidth]{pics/1.5.ubicacion.png}}\end{center}
        \item \textbf{Tabla Evento}
        \begin{center}{\includegraphics[width=0.8\textwidth]{pics/1.6.evento.png}}\end{center}
        \item \textbf{Tabla Cuesta}
        \begin{center}{\includegraphics[width=0.8\textwidth]{pics/1.7.cuesta.png}}\end{center}
        \item \textbf{Tabla Asiste}
        \begin{center}{\includegraphics[width=0.8\textwidth]{pics/1.8.asiste.png}}\end{center}
    \end{itemize}

    Y se comprueba que las tablas se generan correctamente:

    \begin{center}{\includegraphics[width=0.8\textwidth]{pics/1.9.check.png}}\end{center}

    A continuación, se insertan los datos en las diferentes tablas:

    \begin{itemize}
        \item \textbf{Datos de la tabla Actividad}
        \begin{center}{\includegraphics[width=0.8\textwidth]{pics/1.11.actividad.png}}\end{center}
        \item \textbf{Datos de la tabla Artista}
        \begin{center}{\includegraphics[width=0.8\textwidth]{pics/1.12.artista.png}}\end{center}
        \item \textbf{Datos de la tabla Persona}
        \begin{center}{\includegraphics[width=0.8\textwidth]{pics/1.13.persona.png}}\end{center}
        \item \textbf{Datos de la tabla Localidad}
        \begin{center}{\includegraphics[width=0.8\textwidth]{pics/1.14.localidad.png}}\end{center}
        \item \textbf{Datos de la tabla Ubicación}
        \begin{center}{\includegraphics[width=0.8\textwidth]{pics/1.15.ubicacion.png}}\end{center}
        \item \textbf{Datos de la tabla Evento}
        \begin{center}{\includegraphics[width=0.8\textwidth]{pics/1.16.evento.png}}\end{center}
        \item \textbf{Datos de la tabla Cuesta}
        \begin{center}{\includegraphics[width=0.8\textwidth]{pics/1.17.cuesta.png}}\end{center}
        \item \textbf{Datos de la tabla Asiste}
        \begin{center}{\includegraphics[width=0.8\textwidth]{pics/1.18.asiste.png}}\end{center}
    \end{itemize}

    Y se comprueba que los datos se insertan correctamente:

    \begin{center}{\includegraphics[width=0.8\textwidth]{pics/1.19.check.png}}\end{center}

    \subsection{Triggers}    
    Después se crea el trigger que está asociado a la comprobación del aforo:

    \begin{itemize}
        \item \textbf{Trigger - Aforo}
        \begin{center}{\includegraphics[width=0.8\textwidth]{pics/2.1.trigger.png}}\end{center}
    \end{itemize}

    \textbf{Nota}: tal y como se ha creado este trigger, permitiría que, si solo queda 1 hueco, y 2 personas se registran en exactamente el mismo instante, el aforo 
    se superaría. Por tanto, para no complicar demasiado el trigger, se asume que nunca se registrarán personas en el mismo instante. 

    Además, para comprobar el correcto funcionamiento del trigger, se consulta el aforo de las diferentes ubicaciones, y se trata de añadir 1 asistente 
    a aquella ubicación que está en su límite de aforo:
    
    \begin{center}{\includegraphics[width=0.8\textwidth]{pics/2.2.check.png}}\end{center}

    \subsection{Vistas}  
    Para conseguir una mejor eficiencia en las consultas, se crea la siguiente vista asociada a los costes de cada actividad:
    
    \begin{center}{\includegraphics[width=0.8\textwidth]{pics/3.vista.png}}\end{center}
    
    \subsection{Consultas}  
    Y por último se realizan las consultas:

    \begin{center}{\includegraphics[width=0.8\textwidth]{pics/4.1.png}}\end{center}
    \begin{center}{\includegraphics[width=0.8\textwidth]{pics/4.2.png}}\end{center}
    \begin{center}{\includegraphics[width=0.8\textwidth]{pics/4.3.png}}\end{center}
    \begin{center}{\includegraphics[width=0.8\textwidth]{pics/4.4.png}}\end{center}
    \begin{center}{\includegraphics[width=0.8\textwidth]{pics/4.5.png}}\end{center}
    \begin{center}{\includegraphics[width=0.8\textwidth]{pics/4.6.png}}\end{center}
    \begin{center}{\includegraphics[width=0.8\textwidth]{pics/4.7.png}}\end{center}
    \begin{center}{\includegraphics[width=0.8\textwidth]{pics/4.8.png}}\end{center}
    
    En la consulta de eventos con más ceros en su valoración, salen 4 y 2 pues aunque solo hay 2 y 1 asistentes que valoran con 0 respectivamente, 
    ambos eventos están duplicados por lo que esos 3 asistentes valoran 2 veces cada uno con cero, dando lugar a 6 valoraciones con valor 0. Además, 
    mencionar que en este caso no se han tenido en cuenta las fechas, pues se quiere saber, en general, que evento es peor valorado.

    \begin{center}{\includegraphics[width=0.8\textwidth]{pics/4.9.png}}\end{center}
    \begin{center}{\includegraphics[width=0.8\textwidth]{pics/4.10.png}}\end{center}
 

\end{document}