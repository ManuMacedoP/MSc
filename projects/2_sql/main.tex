\documentclass[12pt]{article}
\usepackage[spanish]{babel}
\usepackage[left=2.5cm,top=2.5cm,right=2.5cm,bottom=2.5cm]{geometry}
\usepackage{amsthm,amssymb,amsmath}
\usepackage[spanish]{cleveref}
\usepackage{bbm}
\usepackage[utf8]{inputenc}
\usepackage{mathrsfs}
\usepackage{enumitem}
\usepackage{multirow}
\usepackage{parskip}
\usepackage{graphicx}
\usepackage{booktabs}
\usepackage{float}
\usepackage{longtable}
\usepackage{algorithm}
\usepackage{algpseudocode}
\usepackage{subfig}
\usepackage{appendix}
\usepackage{lscape}




\renewcommand{\baselinestretch}{1.5}
\setlength{\parindent}{12pt}
\begin{document}
\begin{titlepage}
        \centering
        { \bfseries \Large UNIVERSIDAD COMPLUTENSE DE MADRID}
        \vspace{0.6cm} 

        {\bfseries  \Large FACULTAD DE ESTADÍSTICA} 
        \vspace{0.6cm}
        
        {\includegraphics[width=0.9\textwidth]{pics/ucm_logo.png}} 
        \vspace{0.6cm}
        
        {\bfseries \Large TRABAJO DE BASE DE DATOS SQL}
        \vspace{1.75cm}
        
        {\bfseries \LARGE Manuel Macedo Pulido}
        \vspace{6mm} 

        {\large Máster Big Data, Ciencia de Datos e Inteligencia Artificial}
        {\large 2025 }
        \vspace{6mm}

        \thispagestyle{empty}
        
    \end{titlepage}

    \pagenumbering{arabic} 
    \setcounter{page}{1}

    \newpage
    \tableofcontents

    \newpage
    \begin{landscape}
        \section{Diagrama Entidad-Relación}
        {\includegraphics[width=1.4\textwidth]{pics/diagrama.png}} 
    \end{landscape}

    \newpage
    \section{Diseño conceptual}
    En esta sección se definen y explican las entidades del modelo 'Entidad-Relación' para el ejercicio \textit{Empresa de enventos}, 
    así como sus relaciones y las asunciones que se han considerado para construir el modelo. 

    \subsection{Entidades}
    Entidades seleccionadas para el diseño conceptual:

    \begin{itemize}
        \item \textbf{Evento}\\
            Esta entidad es el centro de todo el ejercicio, pues representa los diferentes eventos que la empresa organiza. 
            Y cuenta con los siguientes atributos: \\
                \hspace{0.5cm} -- {\bfseries Nombre}. \\
                \hspace{0.5cm} -- {\bfseries Fecha}: será un TimeStamp que contenga Fecha y hora. \\
                \hspace{0.5cm} -- {\bfseries Precio}: se asume que es fijo para cada evento, pues no se indica lo contrario en el enunciado. 
                Además, por no complicar demasiado el problema, asumimos que cada persona solo puede comprar su propia entrada. 
                De este modo evitamos que se produzcan las reventas de entradas masivas, como pasa hoy en día. \\
                \hspace{0.5cm} -- {\bfseries Descripción}.
            \begin{center}
            {\includegraphics[width=0.5\textwidth]{pics/evento.png}}
            \end{center}

        \item \textbf{Actividad}\\
            Esta entidad representa las actividades que se realizan en cada evento. 
            Y cuenta con los siguientes atributos:\\
                \hspace{0.5cm} -- {\bfseries Nombre}: se asume que el nombre de la actividad es único (e.g. A night of summer magic). \\
                \hspace{0.5cm} -- {\bfseries Tipo}.
            \begin{center}
            {\includegraphics[width=0.3\textwidth]{pics/actividad.png}}
            \end{center}

        \item \textbf{Artista}\\
            Esta entidad representa aquellos artistas que participan en los eventos. 
            Y cuenta con los siguientes atributos: \\
                \hspace{0.5cm} -- {\bfseries Nombre}: se asume que el nombre del artista es único (e.g. Leiva). \\
                \hspace{0.5cm} -- {\bfseries Biografía}.
            \begin{center}
            {\includegraphics[width=0.3\textwidth]{pics/artista.png}}
            \end{center}

        \item \textbf{Persona}\\
            La entidad persona se puede configurar de diferentes formas. En este caso, se decide seguir la siguiente lógica:
            asumiendo que puede haber personas que tengan mismo nombre y apellidos (e.g. hermanos que se llamen igual o incluso 
            desconocidos que se llamen exactamente igual), y considerando que en el enunciado dice "teléfonos de contacto" (por lo que puede haber más de uno),
            asumimos que toda persona que se registre en un evento tendrá que tener un mail, que además será único. 
            Por último, se tendrá en cuenta que el número de personas que asista a cada evento, deberá cumplir la restricción de aforo de cada evento.
            Por lo que cuenta con los siguientes atributos: \\
                \hspace{0.5cm} -- {\bfseries Nombre}: para no complicar demasiado el problema, asumimos que el nombre completo se introduce en realidad en 3 campos. \\
                \hspace{0.5cm} -- {\bfseries Apellido1}. \\
                \hspace{0.5cm} -- {\bfseries Apellido2}. \\
                \hspace{0.5cm} -- {\bfseries Teléfono1}. \\
                \hspace{0.5cm} -- {\bfseries Teléfono2}. \\
                \hspace{0.5cm} -- {\bfseries Mail}.
            \begin{center}
            {\includegraphics[width=0.75\textwidth]{pics/persona.png}}
            \end{center}
        
        \item \textbf{Ubicación}\\
            Por último, se define la entidad ubicación, que tiene algo más de detalle. El nombre de la ubicación puede no ser único, pues podemos tener el mismo
            nombre en diferentes localidades. Es por ello que necesitamos saber de manera independiente tanto la ubicación como la localidad.
            Además, el atributo localidad es compuesto, por lo que creará tabla. E inicialmente, podríamos pensar que sería suficiente con usar el tipo de localidad 
            y su nombre. Sin embargo, en caso de que nos equivocásemos al introducir el valor del nombre en la tabla localidad, también tendríamos que corregirlo
            en la tabla de la entidad ubicación. Por ello, para evitar este tipo de problemas, añadimos el atributo ID localidad, que tendrá valores únicos, 
            y asumimos que no se cometerán fallos al añadir campos nuevos, pues será un ID numérico al uso. 
            Y cuenta con los siguientes atributos: \\
                \hspace{0.5cm} -- {\bfseries Nombre}. \\
                \hspace{0.5cm} -- {\bfseries Localidad}: este a su vez tiene 3 atributos (ID Localidad, Nombre Localidad y Tipo). \\
                \hspace{0.5cm} -- {\bfseries Dirección}: asumimos que la dirección será un único valor que contendrá toda la información necesaria,
                referente a la localidad previamente especificada. \\
                \hspace{0.5cm} -- {\bfseries Aforo}: este atributo será utilizado para tener en cuenta la restricción de aforo. \\
                \hspace{0.5cm} -- {\bfseries Precio Alquiler}. \\
                \hspace{0.5cm} -- {\bfseries Característica}.
            \begin{center}
            {\includegraphics[width=0.75\textwidth]{pics/ubicacion.png}}
            \end{center}

    
    \end{itemize}

    \newpage
    \subsection{Relaciones}
    A continuación se explican las diferentes relaciones que tienen lugar entre las entidades:

    \begin{itemize}
        \item \textbf{Cuesta} -- Relación entre Actividad y Artista\\
            Esta relación es de tipo N:N (por lo que creará tabla), pues varios artistas pueden participar en varias actividades. \\
            La cardinalidad del lado Actividad → Artista es de tipo (1,N), pues el enunciado dice que 'En cada actividad participa uno o varios artistas". \\
            Y la cardinalidad del lado Artista → Actividad es de tipo (1,N), pues el enunciado dice 'de los artistas que los protagonizan', por lo que asumimos
            que solo se registran artistas que van a participar en actividades (y no artistas que no participan en actividades, lo que daría una cardinalidad
            de tipo (0,N)). \\
            Además, tiene el atributo de relación 'dinero', pues entendemos que cada artista tiene un coste diferente para cada actividad.
        
        \item \textbf{Se Realiza} -- Relación entre Evento y Actividad\\
            Esta relación es de tipo 1:1, pues el enunciado dice que 'En un evento solo se realiza una actividad'. Esto implica que las cardinalidades
            de ambos lados sean de tipo (1,1).

        \item \textbf{Asiste} -- Relación entre Evento y Persona\\
            Esta relación es de tipo N:N (por lo que creará tabla), pues varias personas pueden asistir a varios eventos. \\
            La cardinalidad del lado Evento → Persona es de tipo (0,N), pues el enunciado no dice explícitamene que a cada evento tengan que asistir al menos
            una persona. \\
            Y la cardinalidad del lado Persona → Evento es de tipo (1,N), pues el enunciado dice 'tendremos en cuenta los asistentes a los eventos', 
            por lo que asumimos que solo se registran personas que van a asistir a eventos (y no personas que no asisten a eventos, lo que daría una cardinalidad
            de tipo (0,N)). \\
            Además, tiene el atributo de relación 'valoración', ya que la valoración de cada persona a cada evento es independiente.
                
        \item \textbf{Tiene} -- Relación entre Evento y Ubicación\\
            Esta relación es de tipo 1:N, asumiendo que cada evento esté solo en 1 ubicación (e.g. no hay eventos en el medio de 2 ubicaciones)
            y asumiendo que una ubicación pueded tener más de 1 evento. \\
            La cardinalidad del lado Evento → Ubicación es de tipo (1,1), pues asumimos que cada evento tiene 1 ubicación, y que es única. \\
            Y la cardinalidad del lado Ubicación → Evento es de tipo (1,N), ya que según el enunciado, asumimos que solo se registran ubicaciones asociadas 
            a eventos (en caso contrario la cardinalidad sería de tipo (0,N)) y que varios eventos pueden tener lugar en una misma ubicación.

    \end{itemize}

    \newpage
    \subsection{Paso a tabla}

\end{document}